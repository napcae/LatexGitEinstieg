%%Documentclass definition
\documentclass[xcolor=x11names,compress]{beamer} %beamer main class
%\documentclass[handout]{beamer} %for handouts

%%Package defintion
\usepackage{pgfpages}  %Printing multiple pages on one 
\usepackage{graphicx}  %Package 
\usepackage{tikz}      %Package for drawing
\usepackage{listings}
%\usepackage{hyperref}  %urls
\usepackage[utf8]{inputenc} 
\usepackage[ngerman]{babel}

%%
%% Beamer Layout %%%%%%%%%%%%%%%%%%%%%%%%%%%%%%%%%%
\useoutertheme[subsection=false,shadow]{miniframes}
\useinnertheme{default}
\setbeamertemplate{footline}{%
\begin{beamercolorbox}{section in head/foot}
    \color{gray}\vskip2pt~  \insertshorttitle\hfill\insertpagenumber{} %
    of \insertpresentationendpage{} ~\vskip2pt
\end{beamercolorbox}
}
\usefonttheme{serif}
\usepackage{palatino}

\setbeamerfont{title like}{shape=\scshape}
\setbeamerfont{frametitle}{shape=\scshape}

\setbeamercolor*{lower separation line head}{bg=DeepSkyBlue4} 
\setbeamercolor*{normal text}{fg=black,bg=white} 
\setbeamercolor*{alerted text}{fg=red} 
\setbeamercolor*{example text}{fg=black} 
\setbeamercolor*{structure}{fg=black} 
 
\setbeamercolor*{palette tertiary}{fg=black,bg=black!10} 
\setbeamercolor*{palette quaternary}{fg=black,bg=black!10} 

\setbeamertemplate{note page}[plain]

\renewcommand{\(}{\begin{columns}}
\renewcommand{\)}{\end{columns}}
\newcommand{\<}[1]{\begin{column}{#1}}
\renewcommand{\>}{\end{column}}
%%%%%%%%%%%%%%%%%%%%%%%%%%%%%%%%%%%%%%%%%%%%%%%%%%
%%

\title[Latex Beamer Einführung]{Einführung in Beamer\\Wie man eine Präsentation mit \LaTeX \,erstellt}
\author{Chi Trung Nguyen}
\institute{HfTL}
\date{5. März 2014}

\begin{document}

	\begin{frame}
		\titlepage
	\end{frame}

\section{Warum \LaTeX\, Beamer?}
	\subsection{Vorteile}
		\begin{frame}{Vorteile}
			\begin{itemize}
				\item Wiederwendung von bereits erstellten Material
				\pause
				\item Versioncontrol
			\end{itemize}

		\end{frame}

	\subsection{Nachteile}
		\begin{frame}{Nachteile}
			\begin{itemize}
				\item Zeitintensiv zu erlernen
				\pause
				\item Designs vorgegeben
			\end{itemize}
		\end{frame}

\section{Installation}
	\subsection{Itemize}
		\begin{frame}
			\item Windows: MikTex
			\pause
			\item Mac: MacTex
		\end{frame}

\section{Basics}
	\subsection{Itemize}
		\begin{frame}
		\href{http://s.ctnguyen.net/latex_hftl}{http://s.ctnguyen.net/latex\_hftl}
		\end{frame}

\section{?}
	\subsection{asd}
		\begin{frame}
			\href{http://s.ctnguyen.net/latex_hftl}{http://s.ctnguyen.net/latex\_hftl}
		\end{frame}

	\subsection{asd}
		\begin{frame}
			\huge{Fragen?}
		\end{frame}
\end{document}